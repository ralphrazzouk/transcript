\documentclass{article}

\usepackage{mathptmx,fullpage}
\usepackage{amssymb,amsmath,amsthm,amsfonts}
\usepackage{tikz-cd}
\newtheorem{thm}{Theorem}
\newtheorem*{dfn}{Definition}
\newtheorem{cor}[thm]{Corollary}
\newtheorem{lem}[thm]{Lemma}
\newtheorem*{theorem}{Theorem}


\newcommand{\ket}[1]{|#1\rangle}
\newcommand{\bra}[1]{\langle#1|}
\newcommand{\braket}[2]{\langle#1|#2\rangle}
\newcommand{\bbC}{\mathbb{C}}
\newcommand{\calH}{\mathcal{H}}
\newcommand{\Z}{\mathbb{Z}}

\DeclareMathOperator{\poly}{poly}

\begin{document}

\noindent
\fbox{
	\parbox{\linewidth}{
		\vspace{-.3cm}
{\bf \Large \begin{center}
CS 593/MA 592 - Intro to Quantum Computing \\
Spring 2024 \\
Tuesday, February 20 - Lecture 7.1
\end{center}}
\vspace{-.3cm}
	}
}

\vspace{.3cm}

\noindent {\bf Reading:} Appendix 2

\

\noindent{\bf Agenda:}
\begin{enumerate}
	\item Groups
	\item Cosets, Quotient, etc.
	\item Representations
        \item Group algebra and regular representations
\end{enumerate}

\section{Groups}
\textbf{Intuition:} "A group is an abstract symmetry type" \\

\begin{dfn} A group $G$ is a set with a binary operation
	
	\[\begin{aligned}
	\cdot: G \times G &\to G\\
	\end{aligned}\]
that satisfies the following axioms:
\begin{enumerate}
	\item Associativity
	\item There exists an identity element e such that $ge = eg = g$ for all $g \in G$.
	\item There exists an inverse for all  $g \in G$ and  $g^{-1} \in G$ such that $gg^{-1}=g^{-1}g=e$
	
	
\end{enumerate}
\end{dfn}


G is finite if $|G| < \infty$.  We call $|G|$ the \emph{order} of G. The \emph{order} of $g \in G$ is $|g|=min\{k\ge 1 | g^{k} = e\}$.

\vspace{3mm}

\textbf{A subgroup} of G is a subset $H \subseteq G$ such that:
\begin{enumerate}
	\item For all $h \in H$, $h^{-1} \in H$.
	\item For all $h_{1}, h_{2} \in H$, $h_{1}h_{2} \in H$
\end{enumerate}
We write $H \le G$ if it is a subgroup.

Given $x_{1},...,x_{k} \in G$ the sub group generated by them is:

$$\langle x_{1},...,x_{k}\rangle = \bigcap_{H \leq G} H$$.

We call $\langle x\rangle$ the cyclic subgroup generated by $x$, since it consists of all powers of $x$ (positive, negative and 0 powers).

\textbf{Lemma:} If $g \in G$, then $|g|=|\langle g\rangle|$. 

\begin{thm}[Lagrange's Theorem]
If $H \le G$, then $|H|$ divides $|G|$.
\end{thm}

A group G is \textbf{abelian} or commutative, if for all $g_{1}, g_{2} \in G$ we have $g_{1}g_{2} = g_{2}g_{1}$.

\subsection{Examples of Groups}
\subsubsection{($\Z / N\Z,+$)} 
This is the group of addition mod N. It will be of great importance later when $N=2^{n}$.

\subsubsection{($(\Z / 2\Z)^{n},+$)} 
Given $a_{1},.....a_{n}$ and $b_{1},.....b_{n}$, then $(a_{1},.....a_{n})+(b_{1},.....b_{n}) = a_{1}+b_{1},.....,a_{n}b+_{n}$ where the addition is mod N.

\textbf{Fact:} Every finite abelian group is isomorphic to a group of the form $$\bigoplus_{i=1}^{k}\Z / N_{i}\Z$$ where the $N_{i}s$ are positive integers.
Here, note that if A and B are two groups, then $$A\oplus B = \{(a,b) | a \in A, b \in B\}$$ is also a group.
If we apply the Chinese remainder theorem, we can classify finite abelian groups as sums of cyclic groups of prime power order.

\subsubsection{U(d)}
This is the unitary group of $d \times d$ unitary matrices.  Of course $|U(d)| = \infty$ In fact, it is uncountably infinite. Better yet,$U(d)$ is a "Lie group" meaning it is both a grup and smooth manifold, and can be understood rather well using its associated Lie algebra. It also has a subgroup $SU(d) \subseteq U(d)$. \\

$U(d)$ is not abelian unless $d=1$.

\subsubsection{$S_{n}$}
The symmetric group on n elements.  That is, the set of all permutations of the set $\{1,\dots,n\}$:
$$S_{n} = \{F:\{1,...,n\} \to \{1,...,n\} \mid F \text{ is a bijection}\} .$$
$S_{n}$ is not abelian unless $n=2$.

\begin{dfn}
    A \emph{homomorphism} is a function 
    $$\phi:G_{1} \to G_{2}$$
    such that
    $$\phi (xy) = \phi(x)\phi(y)$$
    for all $x, y \in G_1$. If $\phi$ is bijective, then it is called an \emph{isomorphism}.
\end{dfn}

\begin{thm}{(Cayley's Theorem)}
   Let $|G| = n$ and fix an enumeration of the elements of G, $G = \{x_{1},...,x_{n}\}$, for each $g \in G$, define 
   $$L_{g}:\{1,...,n\} \to \{1,...,n\}$$
   where $j$ is the unique index such that $gx_{i} = x_{j}$.  Then $L_{g}$ is a well-defined bijective function. Moreover, the function
\[\begin{aligned}
G &\to S_{n} \\
g &\mapsto L_{g}
\end{aligned}\]
    is an injective group homomorphism.
\end{thm}
Thus, every (finite) group is a subgroup of a permutation group.

\subsection{Cosets, etc.}
Given $H\leq G$ and  $g \in G$, the left H-coset of g is 
$$gh = \{gh|h\in H\}$$

\textbf{Lemma:} $g_{1}H = g_{2}H$ iff there exists $h\in H$ such that $g_{2}=g_{1}h$. \\

The set of all left H-cosets is denoted 
$$G/H = \{gH | g \in G\}$$

\textbf{Note:} $G/H$ is a partition of G in which each part has size $|H|$.  This proves Lagrange's theorem. \\

We can also similarly define right H-cosets. \\

We say H is a \emph{normal} subgroup if for all $g \in G $, $gH=Hg$.  We denote this $H \unlhd G$. \\

\begin{theorem}
The following are equivalent
\begin{itemize}
\item $H \unlhd G$
\item The function $(G/H) \times (G/H) \to G/H$ and $(g_{1}H,g_{2}H) \mapsto (g_{1}g_{2})H$ is well defined and makes $G/H$ a group.  We call $G/H$ with this group operation the quotient group (of $G$ by $H$).
\end{itemize}
\end{theorem}

\textbf{Note} for abelian $G$, all subgroups are normal. \\

If $\phi : G_{1} \to G_{2}$ is a homomorphism, then the kernel is $\ker \phi = \{x\in G_{1}  | \phi(x)=1\}$


\begin{thm}{(First Isomorphism theorem)}

If $\phi : G_{1} \to G_{2}$ is a homomorphism, then $\phi(G_{1}) \leq G_{2}$, $\ker \phi \unlhd G_{1}$ and $\phi(G_1) \cong G_{1}/\ker \phi$.  
   
\end{thm}

\section{Representations}
Let V be a vector space over the complex numbers $\mathbb{C}$. Define the general linear group of V to be
\[ GL(V) = \{F: V\to V \mid F \text{ is linear and bijective}\}.\]
If $V = \mathbb{C}^{n}$ we write $GL(n,\mathbb{C}) = GL(\mathbb{C}^{n})$.

A \emph{representation of a group} $G$ on $V$ is a homomorphism 
$$\rho : G \to GL(V)$$
Suppose 
$$\rho_{1} : G \to GL(V_{1})$$
$$\rho_{2} : G \to GL(V_{2})$$
are two representations. We say $\rho_{1}$ and $\rho_{2}$ are \emph{isomorphic} if there exists an isomorphism of vector spaces
$$\Phi : V_{1} \to V_{2}$$
such that
$$\rho_{2}(g) = \Phi \circ \rho_{2}(g) \circ \Phi^{-1}$$
for all $g \in G$. In other words for all $g \in G$ the following diagram commutes:
\[
\begin{tikzcd}
V_{1} \arrow[r, "\rho(g)"] \arrow[d, "\Phi"'] & V_{1} \arrow[d, "\Phi"] \\
V_{2} \arrow[r, "\rho(g)"']                   & V_{2}                  
\end{tikzcd}
\]
A representation $\rho: G \to GL(v)$ is \emph{unitary} if V is a finite-dimensional Hilbert space and $\rho(G) \subseteq U(V) \subseteq GL(V)$. \\

\textbf{Lemma:} Every representation over $\mathbb{C}$ of a finite group is isomorphic to a unitary representation. \\


\noindent{\bf Goal of representation theory:}
\begin{enumerate}
	\item Classify the representations of a group.
	\item Understand how the representation of G reflects the underlying structure of G.
	
\end{enumerate}

To this end, there are two types of representations we are interested in:

\begin{enumerate}
	\item Faithful:\\ $\rho : G \to GL(v)$ such that $\rho$ is injective.
	\item Irreducible, which we will define momentarily.
	
\end{enumerate}

\textbf{Note:} Neither property implies the other.

A representation $\rho: G \to GL(v)$ is \emph{reducible} if there exists a non-trivial, proper W such that $\rho(g)(W) \subseteq W$ for all $g \in G$. 

An irreducible representation is a representation that is not reducible and not 0-dimensional.  We often call these ``irreps." \\

\textbf{Lemma:} Every 1-dimensional representation is an irrep.\\

In particular, the trivial 1-dimensional representation
\[\begin{aligned}
\rho: G &\to GL(\mathbb{C}) = \mathbb{C}^\times = \mathbb{C} - \{0\} \\
g &\mapsto 1
\end{aligned}\]
is always irreducible.

\begin{dfn}
    A conjugacy class if G is a subset of $C \subseteq G$ such that 
    $$C = \{xgx^{-1} | x \in G\}$$
for some $g \in G$.
\end{dfn}

\begin{thm}
    If $G$ is a finite group, then the number of complex irreps of $G$ (considered up to isomorphism) equals the number of conjugacy classes of $G$.
\end{thm}

The best way to prove this is by using ``character theory".
Given any representation $\rho: G \to GL(v)$, the \emph{character} of $\rho$ is 
\[\begin{aligned}
Tr_\rho: G &\to \mathbb{C} \\
g &\mapsto Tr(\rho(g))
\end{aligned}\]

\textbf{Note:} $Tr_{\rho}$ is not a homomorphism.  (It us a ``class function," meaning it is constant on the conjugacy classes of $G$.)

Moreover, given any rep $\rho: G \to GL(v)$, there exists a unique collection of irreps $\rho_{1},...,\rho_{k}$ (possibly with multiplicities) such that 
\[\rho \cong \bigoplus_{i=1}^{k} \rho_{i}\]

\textbf{Corollary:} If A is abelian, then it has $|A|$ many irreps.\\

\textbf{Lemma:} If A is abelian and $\rho: G \to GL(v)$ is irrep, then $\dim V  = 1$. \\

\textbf{Proof:} By prior lemma, we can assume $\rho$ is unitary. In particular for all $a \in A$, $\rho(a)$ is unitary. However, we know that unitary matrices are diagonalizable. Since A is Abelian, the $\rho(a)$ are simultaneously diagonalizable. Now, let $\beta = \{\Vec{v_{1}},...,\Vec{v_{n}}\}$ be basis for which we have a diagonal representation for each $i = 1,...,n$ in the one-dimensional subspace. Clearly span$\{\Vec{v_{i}}\}$ is invariant under the $A$ action. Because $\rho$ is assumed to be irreducible, we conclude that $\Vec{v_i}$ must span all of $V$.  In other words, $V$ is 1-dimensional, as desired. \\

Thus, the irreps of an abelian group A are the same thing as a homomorphism $A \to U(1)$. \\

\begin{dfn}. If $A$ is a (finite) abelian group, then the \emph{dual group} is the set of all irreducible representations of $A$:
$$\hat{A}:= Hom(A,U(1))$$
The group operation is defined as follows.
Given
$$\rho_{1}: A\to U(1)$$
$$\rho_{2}: A\to U(1)$$
define
\[\begin{aligned}
\rho_{1} \otimes \rho_{2} : A&\to U(1)\\
a &\mapsto \rho_{1}(a)\rho_{2}(a)
\end{aligned}\]
\end{dfn}

\begin{thm}{(Pontryagin duality)}
    Let A be a finite abelian group, then
    \begin{enumerate}
        \item $\otimes$ makes $\hat{A}$ into an abelian group.
        \item $\hat{A} \cong A$ (But NOT naturally)
        \item $\hat{\hat{A}} \cong A$ (Naturally)
    \end{enumerate}
    
\end{thm}



\end{document}